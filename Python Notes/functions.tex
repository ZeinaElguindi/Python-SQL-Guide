\subsection{Function Parameters}
\begin{itemize}
    \item \textbf{Argument} is the value that you pass into a function
    \item \textbf{Parameter} is the variable that the argument is stored in
\end{itemize}

\textbf{Default Values}
A function's arguments can have default values. You use the "=" operator to assign a default value to an argument, which makes passing that parameter to the function optional.
\begin{itemize}
    \item If a function is called without an argument, then the default argument is used
    \item If a function accepts 3 parameters, and of those, 2 are default, then 1 argument must be passed in, and the other 2 are optional.
    \begin{itemize}
        \item If you wish to pass in optional arguments, you must use a keyword to specify which parameter the argument belongs to.\\
        Note: This is not necessary if you are passing in arguments in the order that they are accepted in.
    \end{itemize}
\end{itemize}
Once a function returns, the value stored in the parameter is forgotten.\\
The print() function also contains optional parameters 'end' and 'sep' to specify what should be printed after the argument:
\begin{itemize}
    \item print() adds a new line character at the end of it's argument, but this can be bypassed by including passing in an argument $end = ''$ .
    \item You can use the $sep=','$ argument to separate the strings passed into the print function.
\end{itemize}

\subsection{Return Values}
The return statement is the final and terminating execution of a function
\begin{itemize}
    \item If a value is returned, this value is passed back to the orginal function call statement
    \item If no value is specified for return, the 'None' value is returned
\end{itemize}


\subsection{Local and Global Scope}
Parameters and variables assigned in a function are in that function's\textit{local scope}, such variable is called a \textit{local variable}. Variables assigned outside all functions exist in the \textit{global scope}.
\begin{itemize}
    \item As mentioned previously, when a function returns, all variables assigned within it (local variables) are forgotten.
    \item Variables in different scopes can have the same name as they are able to store different values within their scopes.
\end{itemize}
Global variables can be used in the local scope, although the reverse is not true.\\
If you would like to use/alter a global variable within a local scope, declare the global statement: $global varname$ within the function. The program will now know that varname refers to the global variable, and there is no need to create a new local variable with that name.

\subsection{Packing and Unpacking Sequences}
\textbf{Automatic Packing}\\
If a series is comma-separated, then it automatically gets formatted to a tuple when assigned.
\[even = 2,4,6 \iff even = (2,4,6)\]

Although you cannot have more than one return statement in a function (unless through the use of logic statements), you can return more than one value within the same return statement:
\[return \hspace{0.1cm}x,y\]
This will return a tuple with the values of x and y as items.
\\[0.35cm]
\textbf{Automatic Unpacking}\\
Python can also automatically unpack iterable objects (lists, tuples, etc).
\[even = 2,4,6\]
This can also be used with loops:
\begin{center}
    \begin{verbatim}
        for x,y in [(1,3), (7,8)]
        for k,v in animals.items()
    \end{verbatim}
\end{center}
The first loop will assign x and y to each value within the tuple.
The second loop will iterate over the key and value of the animals dictionary items.

\subsubsection{Simultaneous Assignment}
\[x,y,z = [1,2,3]\]
\[x = 1\]
\[y = 2\]
\[z = 3\]

This feature is especially useful for variable swapping:
\[x,y = y,x\]

The variable 'x' will be assigned to the old value of y, and the variable 'y' will be assigned to the old value of x.\\
This swapping method is more efficient than the following:
\[temp = x\]
\[x= y\]
\[y= temp\]