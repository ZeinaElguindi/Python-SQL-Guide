

\section{Series}
A series is a one-dimensional array. 
\begin{itemize}
    \item A series is essentially a column in a spreadsheet
    \item The series axis is the \textit{index}
\end{itemize}



\subsection{Creating a Series}

\begin{center}
    \textbf{series = pd.Series( data, index, dtype)}
\end{center}

\begin{center}
    \begin{tabular}{c|c|c}
       Argument  & Type & Description  \\
       \hline
         data & list / tuple & Data to be displayed within the column.\\
         index \textit{(optional)}& list / tuple & Index labels. Default: Integer Indexing.\\
         dtype \textit{(optional)} & data type  & Data type specification. Default: Inferred.
    \end{tabular}
\end{center}

\subsection{Accessing Elements in a Series}
To access an element(s), use the [ ] operator, with the index (type: int) that you would like to access.


\begin{center}
\begin{minipage}{0.45\textwidth}
        Example 1: \\
        \\[0.01cm]
        data = [1,2,3]\\
        ser = pd.Series(data)\\
        print(ser[2])\\
        Output: 3
\end{minipage}
\hspace{0.05\textwidth} % Space between the minipages
\begin{minipage}{0.45\textwidth}
    Example 2:\\
    \\[0.01cm]
    print(ser[:1])\\
    Output:\\
    \begin{tabular}{c|c}
       0  &  1\\
        1 & 2
    \end{tabular}
    
\end{minipage}
\end{center}


\subsection{Indexing and Selecting Data}
\begin{center}
    \begin{tabular}{c|c|c}
        Operator/Function & Name & Purpose\\
        \hline
        [ ] &  Indexing Operator\\
         .loc[]&  Label Location & Selecting data by index label\\
         .iloc[] & Integer Location & Selecting data by integer index\\
    \end{tabular}
\end{center}

Note: Using series, there are many similarities between these 3. .loc[] uses the index names to select data and is therefore inclusive, while .iloc[] uses index positions, and is therefore end-exclusive.






\section{DataFrames}

A dataframe is a two-dimensional data structure. The datasets within it are stored in 2 axes, the rows and columns (essentially a table)


\subsection{}{Creating a DataFrame}
\subsubsection{From an aaray}
\begin{minted}{python}
    arr = ['Physics', 'Chemistry', 'Biology']
    df = pd.DataFrame(arr)
\end{minted}

The output of this would be a table, with one column. Each row represents the different list elements, and the index and column labels represent the row and column headers respectively.

\subsubsection{From a Dictionary}





\begin{minted}{python}
    hi
\end{minted}














\subsection{Viewing Data}

\subsection{Selecting Data}