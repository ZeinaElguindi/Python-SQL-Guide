\subsection{Raising an Exception}
You can raise exceptions to validate user input or parameter types.\\
Checking the validity of a parameter for a division function may look like the following:
\begin{verbatim}
    def divide(dividend, divisor):
        if divisor == 0:
            raise ValueError("Divisor cannot be zero.")
        return dividend / divisor
\end{verbatim}

When an exception is raised, a new instance of the Error class is created, with the error message serving as a parameter for the constructor \textcolor{darkgray}{(chapter: \ref{sec: class components} for more details)}.

You can raise built-in exceptions such as 'ValueError' or 'TypeError', or you can raise an exception from a custom exception class.
\begin{verbatim}
    class CustomError(message):
        def __init__(self, message):   # constructor method
            self.message = message
            super().__init__(self.message)  # calls constructor of Exception class
    def sqrt(x):
        if x < 0:
            raise CustomerError ('x must be greater than 0.')
    try:
        print(sqrt(-3))
    except CustomError as error:
        print(f'Error: {error}
\end{verbatim}

\subsection{Catching an Exception}
To avoid the chances of an exception being raised, put code that may have an error in a \textit{try} clause. Once an error is detected, the program will move to the \textit{except} clause.
\begin{itemize}
    \item The program will move to the except clause once an error is raise, and it will not return to the try clause.
\end{itemize}

\begin{verbatim}
    try:
        file = open('sample_file.txt')
    except IOError:
        print('Invalid file path')
\end{verbatim}

\subsubsection{Example:}
\begin{verbatim}
    def divide(divisor):
        try:
            return 30/divisor
        except ZeroDivisionError:
            return 'Division by 0 is not possible.'
    print(divide(3))  # output: 10
    print(divide(0))  # output: 'Division by 0 is not possible.'
    print(divide(10)) # output: 3
\end{verbatim}
Note: In this case, the final line gets executed, although if the \textit{try} clause was surrounding the print statements, '3' would not be outputted as the program will move to the \textit{except} clause and not return.