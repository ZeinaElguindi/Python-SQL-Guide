\subsection{Classes Overview}
A class serves as a user-defined template used to create objects.
A class has defined attributes and methods (functions) that its created objects will have.
\begin{itemize}
    \item Provides a set of behaviors for instances of the class (objects) to follow
    \item Defines the properties of the class's objects
\end{itemize}


\subsection{Components of a Class}
\label{sec: class components}
\textbf{Attributes}\\
Attributes are variables/properties of the class or its instances.\\
For example, a 'Student' class may have attributes of name, year and program
\begin{verbatim}
    student.name = Jack Hemington
    student.year = 4
    student.program = Economics
\end{verbatim}

\textbf{Methods}\\
Methods are defined functions/behaviors that objects in the class can perform.
For example, the 'Student' class may have a method called study, which makes the student study.
\begin{verbatim}
    def study(self):
        print(f'{self.name} is now studying!')
\end{verbatim}

\textbf{Constructor Method}\\
The constructor method instantiates an object, a.k.a, it initializes the attributes of the class.\\
When a new instance of the class (object) is created, the constructor method is automatically called to assign the arguments and return an initialized object.
\begin{verbatim}
    class Student:
    def __init__(self, name, year, program):
        self.name = name
        self.year = year
        self.program = program

    student1 = Student('Jack Hemington', 4, 'Econonomics')
\end{verbatim}

\subsection{Creating a Class}
\textbf{Defining a Class}
\begin{itemize}
    \item Define the class
    \item Set up the constructor method and attributes
    \item Define a method for the class
\end{itemize}

Example:
\begin{verbatim}
    class Student:
        def __init__(self, name, year, program):
        self.name = name
        self.year = year
        self.program = program

        def study(self):
            print(f'{self.name} is now studying!')
\end{verbatim}
\vspace{0.3cm}
\textbf{Instantiating a Class}
To create an object/instance of the class, you call the class as if it were a function, with the necessary values to pass in.
Example:
\begin{verbatim}
    student1 = Student('Jack Hemington', '4', 'Economics')
\end{verbatim}
The constructor method will then be called to initialize this object.
\\[0.5cm]
\textbf{Accessing Attributes and Methods}\\
\textbf{\textcolor{gray}{Accessing Attributes:}}\\
\begin{center}
    objectName.attributeName
\end{center}

\textbf{\textcolor{gray}{Accessing Methods:}}
\begin{center}
    objectName.methodName
\end{center}
\subsection{Complete Example:}
\begin{verbatim}
    class Student:  # define the class

        def __init__(self, name, year, program):  # define constructor method
            self.name = name
            self.year = year
            self.program = program
    
        def study(self):   # define method
            return f'{self.name} is now studying!' # access attribute

    # create instances of the 'Student' class
    student1 = Student("Andy", 4, "Arts and Literature")
    student2 = Student("Ben", 3, "Marketing")

    # call method and access attributes
    print(student1.study())   # calls 'study' method
    print(student1.program)   # output: "Arts and Literature"
    print(student2.year)      # output: 3
    
\end{verbatim}

\subsection{Purpose of "Self"}
The self parameter is used to access variables that belong to the class. 
\begin{itemize}
    \item It must be the first parameter in any methods defined within the class.
\end{itemize}