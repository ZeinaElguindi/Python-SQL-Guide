The parameters of statistical functions must be a single column or expression.\\
The output of these functions is a single column per function.\\
You can rename the outputted column by using an alias.


Note: To filter for columns with an aggregate function applied, you must use the \textcolor{textgray}{\textbf{HAVING}} clause \textcolor{darkgray}{(\ref{sec: having clause})}.

\subsection{Count}
\begin{center}
    \begin{tabular}{c|c}
        Parameter & Description \\
        \hline
        COUNT $(*)$ & Count all rows in the table\\
        COUNT $(column\_name)$ & Count all non-null rows in the column\\
        COUNT $(DISTINCT \hspace{0.15cm}column\_name)$ & Count unique values in the column, excluding NULL rows.
        \\[0.45cm]

    \end{tabular}
\end{center}

Using the \textcolor{textgray}{\textbf{GROUP BY}} clause is especially useful when paired with the COUNT function:\\[0.3cm]
From \textcolor{darkgray}{(\ref{sec: group by})}, we can see that a 'Total' column returned (using the count function), containing the count of each Pet and Breed combination.\\[0.3cm]
Note: It is required to have any non-aggregate function stated within the \textcolor{textgray}{\textbf{SELECT}} statement to be stated within the \textcolor{textgray}{\textbf{GROUP BY}} statement


\subsection{Avg \& Sum}
The \textbf{AVG()} function calculates and returns the average value or a set/column.

\begin{center}
    AVG ([ALL | DiSTINCT] expression)
\end{center}


The \textbf{SUM()} function calculates and returns the average value or a set/column.

\begin{center}
    SUM ([ALL | DiSTINCT] expression)
\end{center}

By default, AVG() and SUM() calculate using all values in the column, although, using the DISTINCT keyword would result in the calculation of unique values only.\\
The values within the column \textbf{must} be numeric.


\subsection{Max \& Min}
The \textbf{MAX()} function returns the maximum value of a set/column.

\begin{center}
    MAX (expression)
\end{center}

The \textbf{MIN()} function returns the minimum value of a set/column.

\begin{center}
    MIN (expression)
\end{center}

The \textcolor{textgray}{\textbf{DISTINCT}} option is not available for MAX and MIN functions.

