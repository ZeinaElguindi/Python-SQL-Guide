
\subsection{Ordering Rows}
"ORDER BY" keyword sorts the rows returned by \textcolor{textgray}{\textbf{SELECT}} clause in either ascending or descending order.


You can also sort rows by giving sortation priorities to specific columns. The following will sort column\_1 in an ascending order, and then column\_1 in descending order.

\begin{minted}[bgcolor=shadedgray]{sql}
ORDER BY 
    column_1 ASC, 
    column_2 DESC;
\end{minted}

\subsection{Grouping Rows}
\label{sec: group by}
The "GROUP BY" keyword groups rows based on the values of 1 or more columns.
\begin{itemize}
    \item This is useful for aggregating data by grouping rows that share common values.
\end{itemize}

The following query counts the number of records for each Pet and Breed in vet\_table. The \textcolor{textgray}{\textbf{GROUP BY}} clause aggregates the count in the 'Total' column. 
\begin{center}
\begin{minipage}{0.45\textwidth}
    \centering
    \begin{tabular}{c|c|c}
     Pet&Breed&VisitReason  \\
     \hline
     Cat&Siamese& Check-up\\
     Cat&Siamese& Vaccination\\
     Dog & Poodle & Check-up
    \end{tabular}
    \captionof{table}{vet\_table}
\end{minipage}
\hspace{0.05\textwidth} % Space between the minipages
\begin{minipage}{0.45\textwidth}
    \centering
    \begin{tabular}{|c|c|c|}
    \hline
    Pet & Breed & Total \\
    \hline
    Dog & Poodle & 1 \\
    Cat & Siamese & 2 \\
    \hline
    \end{tabular}
    \captionof{table}{Query Result}
\end{minipage}
\end{center}
\begin{minted}[bgcolor=shadedgray]{sql}
    SELECT Pet, Breed, COUNT(*) AS Total
    FROM vet_table
    GROUP BY Pet, Breed
    ORDER BY Pet DESC;
\end{minted}
You can now use \textcolor{textgray}{\textbf{HAVING}} to place conditions on the aggregate function 'COUNT(*)' \textcolor{darkgray}{(\ref{sec: having clause})}.